% Chapter 3

\newcommand{\chaptertitle}{Typographie}

\chapter{\chaptertitle} % Main chapter title

\label{Typographie} % For referencing the chapter elsewhere, use \ref{Chapter1}

\lhead{\chaptername{} \thechapter{} - \emph{\chaptertitle}} % This is for the header on each page - perhaps a shortened title

%----------------------------------------------------------------------------------------

Das Kapitel Typographie könnte als eines der wichtigsten Kapitel dieses Projektes beschreiben werden. Typographie kommt in fast jeder Art von Artefakt vor und bildet den Grundstein einer guten Gestaltung. Gerade weil Typographie aber in so vielen Gebieten Anwendung findet, kann hier unmöglich jeder dieser Anwendungsfälle abgedeckt werden.
Das Projekt befasst sich deshalb auf einem grundlegenden Level mit der Typographie und bietet einen Leitfaden zur Erstellung von Fließtexten. Ziel soll es sein, am mit Hilfe des Tools einen gut gesetzten und lesbaren Text zu erstellen.
Sehr Graphische Anwendungsfälle wie zum Beispiel Plakate werden hier bewusst nicht angesprochen.

\section{Schriftarten}
Um einen Text zu setzen müssen eine Reihe von Entscheidungen getroffen werden. Die erste dieser Entscheidungen stellt die Entscheidung über die Schriftart dar.
Schriftarten werden in Kategorien unterteilt, die nicht immer einheitlich sind. So verwendet Strizver \cite{strizver2014type} andere Gruppierungen als  das “British Standards Classification of Typefaces” \cite[S. 51]{baines2005type}. Die zwei Gruppierungen “serif” und “serifenlos” finden sich jedoch in jeder Art der Gruppierung wieder und diese sollen auch in diesem Teil des Projektes betrachtet werden.
Auch andere Schriftarten haben ihre Daseinsberechtigung und Anwendungsfälle, jedoch werden diese unter der Zielsetzung, einen gut lesbaren Fließtext zu erzeugen hier nicht behandelt.

\subsection{Serif oder Serifenlos?}
Je nach Medium lassen sich in der Lesegeschwindigkeit bei serifen und serifenlosen Schriftarten Unterschiede feststellen, so kommen sowohl Josephson __[Josephson, S. (2008). Keeping your readers' eyes on the screen: An eye-tracking study comparing sans serif and serif typefaces.
Visual communication quarterly, 15(1-2), 67-79.]__ als auch Dogusoy, Cicek & Cagiltay __[Dogusoy, B., Cicek, F., & Cagiltay, K. (2016, July). How Serif and Sans Serif Typefaces Influence Reading on Screen: An Eye Tracking Study. In International Conference of Design, User Experience, and Usability (pp. 578-586).
Springer International Publishing.]__ zu dem Ergebnis, dass serifenlosen Schriftarten an Computerbildschirmen besser gelesen werden können. Serife Schriftarten lassen sich hingegen auf Papier (für das sie ursprünglich entwickelt wurden) besser lesen.
Anzumerken ist hierbei jedoch, dass ein gut gesetzter Text in jeder der beiden Schriftarten auch gut lesbar ist.

Weiterhin hängt die Lesegeschwindigkeit außerdem davon ab, ob der Leser die Schriftfamilie bereits kennt und ob diese explizit für das verwendete Medium entwickelt wurde. __[Josephson, S. (2008). Keeping your readers' eyes on the screen: An eye-tracking study comparing sans serif and serif typefaces.
Visual communication quarterly, 15(1-2), 67-79.]__


%----------------------------------------------------------------------------------------

\section{Schriftfamilien}
Wie oben bereits erwähnt gibt es für die Wahl der Schriftfamilie einige Richtlinien, jedoch gibt es keine per se “guten” oder “schlechten” Schriftfamilien.
Wenn möglich empfiehlt es sich, einer für das Zielmedium gestaltete Schriftfamilie zu verwenden, wenn die Schnittstelle des Zielmediums ein Bildschirm irgendeiner Art ist.

Es ist weiterhin empfehlenswert, eine Schriftart zu verwenden, die dem Nutzer bereits bekannt ist. Die Webseite [CSS Font Stack](http://www.cssfontstack.com/) führt eine Liste mit der Verfügbarkeit von Schriftfamilien auf verschiedenen Betriebssystemen, die einen möglichen Orientierungspunkt darstellt. Nachfolgend ein Auszug aus dieser Liste:

__Sans-Serif__
* Arial (Win: 99.84%, MacOS: 98.74%)
* Tahoma (Win: 99.95%, MacOS: 91.71%)
* TrebuchetMS(99.67%, MacOS: 97.12%)
* Verdana (Win: 99.84%, MacOS: 99.1%)
* Fira Sans ([Google Fonts](https://www.google.com/fonts/specimen/Fira+Sans))

__Serif__
* Georgia (Win: 99.4%, MacOS: 97.48%)
* Palatino (Win: 99.29%, MacOS: 86.13%)
* Times New Roman (Win: 99.67%, MacOS: 97.48%)

__Monospace__
* Courier New (Win: 99.73%, MacOS: 95.68%)
* Source Code Pro ([SIL Open Font License](https://github.com/adobe-fonts/source-code-pro)

\subsection{Android}
Die Material Design Guidelines definieren die Schriftfamilien “Roboto” und “Noto” als Schriftfamilien für die Plattform. Diese sollten für den Nutzer, der ein Projekt auf dem Betriebssystem Android entwickelt, hervorgehoben werden.


\subsection{iOS}
Apple spricht sich für die Verwendung der Systemfont “San Francisco” aus. Auch diese sollte einem Nutzer, der für diese Plattform entwickelt hervorgehoben werden.

\subsection{Mischen von Schriftfamilien}
Beim Mischen von verschiedenen Schriftfamilien entstehen häufig unschöne Textsätze. Beim mischen von Schriftfamilien muss unter anderem darauf geachtet werden, dass die beiden Schriftfamiline miteinander harmonisieren, und sie sich deutlich genug unterscheiden.
Die Beurteilung, ob zwei Schriftfamilien gemischt werden sollten  findet auf einer sehr subjektiven Ebene statt, daher wird diese im Tool keine Anwendung finden und sich an den Ratschlag von Ellen Lupon halten:
> Combining typefaces is like making a salad. Start with a small number of elements representing different colors, tastes, and textures. __[Lupton, E. (2014). Thinking with type. Chronicle Books.]__

%----------------------------------------------------------------------------------------

\section{Schriftgrößen}
Wichtig für die Lesbarkeit eines Textes ist neben der Wahl der Schriftart- und Familie vor allem die Größe des Textes. Auch hier gibt es nicht die eine, richtige Größe, die Wahl der Größe hängt wie alle anderen Bereich stark von dem Medium ab, in dem die Texte gelesen werden.
Einige Eingrenzungen lassen sich dennoch finden, so empfiehlt Runk __[Claudia Runk. Grundkurs Typografie und Layout: Für Ausbildung und Praxis (Galileo Design). Galileo Design, 3 edition, 10 2011.]__ für Bildschirme ab 15” eine Schriftgröße von 14px - 16px, Benedikt Lehnert __[http://www.typogui.de/]__ spricht von 14px -24px für Bildschirme und 9 - 12pt für gedruckte Materialien.

Zwar muss die Schriftgröße immer noch individuell angepasst werden, jedoch lassen sich mit diesen Werten bestimmte Grenzen finden. So sind 36px für einen Fließtext im Web wahrscheinlich zu groß und 4pt für einen gedruckten Text zu klein.

\subsection{Überschriften}
Semantisch betrachtet leiten Überschriften einen neuen Sinnabschnitt in einem Text ein. Damit eine Überschrift ihren Zweck erfüllt muss sie einige Eigenschaften besitzen:
Es muss deutlich werden, dass mit der Überschrift ein neuer Abschnitt beginnt, sie muss also aus dem normalen Textfluss heraus fallen. Weiterhin muss deutlich werden, zu welchem Abschnitt die Überschrift gehört.

Um die Überschrift aus dem normalen Textfluss hervor zu heben steht sie klassisch in einer eigenen Zeile. Weiterhin sind sie in der Regel größer als der Fließtext oder (meist bei Überschriften niedrigerer Ordnung) in einem anderen Schriftschnitt gesetzt.

Voraussetzung zum finden einer passenden Überschriftengröße ist also zunächst, dass sie größer sein muss als der Fließtext. Um die Größe der Überschriften zu errechnen bieten sich verschiedene Methoden an.

\subsubsection{Der goldene Schnitt}
Der goldene Schnitt findet in viele Bereichen der visuellen Gestaltung und auch der Natur Anwendung. Er beschreibt ein Verhältnis, das von Menschen in der Regel als harmonisch wahrgenommen wird.
Laut __[Livio, M. (2003). The golden ratio (p. 4). Aryeh Nir Publishers Limited.]__ wird dieses Verhältnis durch eine unendliche, sich niemals wiederholende Zahl beschreiben. Diese wird im Folgenden mit 1.62 angenähert.

Als erster Ansatz bietet es sich an, die Größe einer Überschrift zu errechnen, indem die Schriftgröße des Fließtextes mit dem goldenen Schnitt multipliziert wird.
Für einen Fließtext mit einer Schriftgröße von 14px würde sich eine Überschriftengröße von `14px * 1.62 = 22.68px` ergeben.
Als erster Ansatz scheint diese Methode valide, jedoch bleiben einige Probleme bestehen.

Zum Einen ist 22.68px als Schriftgröße nicht sehr schön. Die Zahl ließe sich aufrunden, aber auch 23px sind als Schriftgröße eher unüblich.
Zum Anderen ist mit dieser Methode zwar die Größe für eine Überschrift gefunden, häufig bestehen Texte jedoch aus mehreren Überschriften verschiedener Ordnung. Hier müssen also noch weiter Werte gefunden werden.

\subsubsection{Typographic Scale}
Die Typographic Scale rührt aus den frühen Zeiten des Buchdruckes, als Texte noch aus einzelnen Buchstaben zusammen gesetzt wurden. Die Auswahl an Schriftgrößen war zu dieser Zeit aus rein technischen Gründen limitiert. Eine Variation diese Typographic Scale kann Abb.X entnommen werden.

__[HIER BILD VON SCALE AUS BUCH]__

Die Typographic scale könnte also im ersten Schritt genutzt werden, um die mit dem goldenen Schnitt errechnete Zahl zu runden.
Die im obigen Beispiel errechnete Zahl lautet 22.68px. Auf der Typographic Scale liegt sie zwischen den Werten 21 und 24. Da er näher an der 24 liegt wir 24px als Schriftgröße für die Überschrift gewählt. Das Ergebnis im gesetzten Text zeigt Abb.X

__[HIER ERSTES BILD VON GESETZTEM TEXT]__

Das Problem der Überschriften verschiedener Ordnungen ist damit jedoch immer noch nicht gelöst. Jedoch kann zumindest eine begrenzte Menge von Zwischenüberschriften aus den Zwischenschritten von Fleißtext und der oben errechneten Überschrift gebildet werden. Im Fall des obigen Beispiels wären diese Überschriften 21px, 18px und 16px groß. Ein entsprechend gesetzter Text findet sich in Abb.X.

__[HIER ZWEITES BILD VON GESETZTEM TEXT]__

Befinden sich eine Überschrift und der Text nur 1px oder 2px auseinander, so scheint es ratsam zu sein, die Überschrift auch anderweitig abzuhaben, etwa durch einen fetten Schriftschnitt.

Ein letztes Problem wird deutlich, wenn man die aufgestellte Gleichung für nahe aneinander liegende Schriftgrößen verwendet. So wäre die Überschrift erste Ordnung für Fließtexte der Größe 14px, 16px und 18px jeweils 21px (s. Rechnung)

```
14px * 1.62 = 22,68px => 24px
16px * 1,62 = 25,92px => 24px
18px * 1,62 = 29,16px => 24px
21px * 1,62 = 34,02px => 36px
```

Um die Überschriften weiter zu differenzieren word eine Konstante mit in die Rechnung eibezogen, die zum Wert des goldenen Schnitts addiert wird. Diese errechnet sich aus der Abweichung der Größe des Fließtextes von 14px geteilt durch 10.
Dieser Wert hat keine weiteren wissenschaftlichen Belege, vielmehr liefert er für Schriftgrößen zwischen 14px und 24px gut gestaffelte Werte und wird daher verwendet.

Eine Berechnung unter Berücksichtigung der Konstante kann Abb.X entnommen werden.

```
14px * 1.62 = 22,68px => 24px
16px * (1,62 + 0,2) = 29,12px => 24px
18px * (1,62 + 0,4) = 36,36px => 36px
21px * (1,62 + 0,7) = 48,72px => 48px
```

Somit wurde ein verwendbarer Ansatz zur Berechnung von Größen von Überschriften basierend auf dem Fließtext gefunden. Dieser Ansatz liefert nur Richtwerte und muss ggf. angepasst werden, weiterhin funktioniert er nur für einen bestimmten Bereich an Schriftgrößen. Für die Verwendung als im Rahmen dieses Tools ist er jedoch ausreichend.


%----------------------------------------------------------------------------------------

\section{Abstände im Text}
TBD

\subsection{Zeilenhöhe und -länge}
Für Zeilenhöhe und -länge lassen sich die genausten Richtwerte finden. Im Print-Bereich wird ein Zeilenabstand von 120 [Run11, S. 150] als optimal angesehen, das W3C empfiehlt, mit Blick auf Menschen mit Sehbehinderungen, einen Zeilenabstand von 150 bis 200 [W3C]. Für die Zeilenlänge legen Studien 65 - 75 CPL (Characters per Line) nahe [BFH02], die einen guten Lesefluss ermöglichen, ohne die Suche nach dem Anfang der nächsten Zeile zu kompliziert zu machen.
Die Toleranz für die Zeilenhöhe im Tool sollte also bei einem Wert zwischen 120 und 160 liegen. Wie hoch genau eine optimale Zeilenhöhe ist, hängt dabei auch von der Schriftart und dem Einsatzgebiet ab, auch hier kann also nur eine Empfehlung abgegeben werden.

\subsection{Absätze und Überschriften}
Ein Absatz beschreibt einen neuen Sinnabschnitt im Text, der auch visuell erkennbar sein muss. In der Regel werden zwei Absätze dabei durch eine Leerzeile voneinander getrennt. In Textsatzprogrammen reicht diese Richtlinie schon aus und ist selbstverständlich, beim Verwenden von Grafikprogrammen oder beim erstellen von Websites kann aber eine menualle Einstellung nötig sein. Die Höhe einer Leerzeile lässt sich durch die Zeilenhöhe berechen:

```
Body: 14px
Line Height: 14px * 140% = 19,6 ~ 20px
Leerzeile: 14px + 20px = 34px
```

Auch die Abstände der Überschriften lassen sich auf Grundlage der Zeilenhöhe berechnen. Als Grundsatz gilt dabei, dass eine Überschrift näher an dem Absatz platziert sein sollte, zu dem sie gehört und dass Überschriften höherer Ordnung mehr Platz haben.

Behalten wir die Größe und Zeilenhöhe des vorherigen Beispiels bei, erhalten wir als Vielfache der Zeilenhöhe 40px, 60px und 80px Spielraum. Diese können nun als obere und untere Abstände der Überschriften verwendet werden:

```
H1: 24px, Top 50px, Bottom 30px (LH * 4)
H2: 21px, Top 40px, Bottom 20px (LH * 3)
H3: 18px Bold, Top 30px, Bottom 10px (LH * 2)
```

Das nachfolgende Bild zeigt einen mit den errechneten Abständen gesetzten Text:

__[HIER NOCH BILD]__

Hier ist erwähnenswert, dass der Großteil aller Programme diese Abstände auch automatisch setzen, wenn die Entesprechenden Elemente auch korrekt ausgezeichnet werden. Wann immer möglich sollte dem Nutzer also zu dieser Methode geraten werden, um die Anzahl von möglichen Fehlerquellen so gering wie möglich zu halten.


%----------------------------------------------------------------------------------------

\section{Kontrast}
Zuletzt sollte im Tool auch der Kontrast eines Textes behandelt werden, der einen großen Einfluss auf die Lesbarkeit eines Textes hat.
Nach Möglichkeit sollte es vermieden werden, lange Textpassagen auch einem Farbigen Hintergrund zu platzieren.
Der sicherst Weg wäre ist hier schwarzen Text auf einem weissen Hintergrund. Auch die umgekehrte Zusammensetzung (also Schwarzer Hintergrund und Weißer Text) ist relativ unproblematisch.

Bei allen weiteren Kombination muss der Kontrast überprüft werden. Glücklicherweise ist dies einer der wenigen Bereiche, in denen konkrete Berechnungen eine Kombination als gut oder schlecht einstufen können.

\subsection{Berechnung}
Zur Berechnung werden die in Abschnitt 1.4.3 der Web Content Accessibility Guidelines des w3c definierten Berechnungen verwendet. Diese unterteilen die Kontraste von zwei Farben in drei Level, A (3:1), AA(4.5:1) und AAA(7:1). Das Tool übernimmt diese Einteilung, eine Warnung wird ausgegeben, sobald der Wert für den Kontrast das definierte Level AA unterschreitet.

Anbei findet sich die Umseztung der in (https://www.w3.org/TR/2016/NOTE-WCAG20-TECHS-20161007/G18) definierten Rechnungen in JavaScript, wie sie auch im PoC vorhanden ist.

%----------------------------------------------------------------------------------------

\clearpage
