% Chapter 5

\chapter{Whitespace} % Main chapter title

\label{Whitespace} % For referencing the chapter elsewhere, use \ref{Chapter1}

\lhead{\chaptername{} \thechapter{} - \emph{Whitespace}} % This is for the header on each page - perhaps a shortened title

%----------------------------------------------------------------------------------------

Bei den Recherchen im Bereich Whitespace wurde schnell deutlich, dass dieses Gebiet, anders als beispielsweise die Typographie, deutlich weniger konkrete Regeln bietet. Viele der Entscheidungen können nur sehr individuell, von Gestaltung zu Gestaltung, getroffen werden. \\
Die Hauptfrage in diesem Abschnitt ist also zunächst: Gibt es überhaupt konkrete Regeln, auf dessen Basis das Tool arbeiten könnte?

%----------------------------------------------------------------------------------------

\section{Was ist Whitespace?}

Whitespace (auch \textit{negative space} oder \textit{Weißraum}) ist leerer Raum innerhalb einer Gestaltung. Für die Qualität einer Gestaltung spielt dieser eine zentrale Rolle:

\begin{quote}
The single most overlooked element in visual design is emptiness. The lack of attention it receives explains the abundance of ugly and unread design. [...]
Design elements are always viewed in relation to their surroundings. Emptiness in two-dimensional design is called white space and lies behind the type and imagery. But it is more than just the background of a design, for if a design's background alone were properly constructed, the overall design would immediately double in clarity and usefulness. Thus, when it is used intriguingly, white space becomes foreground. The emptiness becomes a positive shape and the positive and negative areas become intricately linked. \cite[S.19]{white2011elements}
\end{quote}

Whitespace ist also kein Raum, den es noch zu füllen gilt, sondern bewusst leer gelassener Raum, dessen Aufgabe es ist, Elemente in ihrer Wichtigkeit zu unterstützen. \\
Eine erste, recht ungenaue Regel die sich aufstellen lässt ist die, dass wichtige Elemente mehr Weissraum um sich haben als weniger wichtige. Diese Regel wurde beispielsweise in der Typographie angewendet: Überschriften höherer Ordnung haben im Gegensatz zu denen niedrigerer Ordnung größere Abstände zum Text.

Weiterhin lässt sich feststellen, dass eine Gruppe von Elementen, die dem Gesetz der Nähe \cite{mayer2005einfuhrung} unterliegt, zueinander weniger Whitespace aufweist, als nach außen.

Genaue Werte, wie groß Abstände im Verhältnis zu Elementen sein müssen, um ästhetisch ansprechend zu wirken, lassen sich jedoch nicht finden. Auch Konsistenzen im Weißraum sind schwer zu definieren. So haben Elemente häufig andere Horizontale als Vertikale Abstände, oft sind selbst die Abstände nach oben und unten verschieden. \\
Das Tool kann an dieser Stelle also nur grobe Regeln im Umgang mit Whitespace vermitteln. Eine Interaktivität ließe sich zum Beispiel in der Gruppierung von Elementen oder dem Bilden einer Hierarchie mittels Whitespace schaffen. Weiterreichende Möglichkeiten zur interaktiven Vermittlung von Inhalten konnten nicht gefunden werden. Das Kapitel Whitespace ist somit das Themengebiet im Projekt, dass die wenigsten verwertbaren Ergebnisse lieferte.

Offen bleibt die Frage, ob mit der Analyse eines ausreichend großen Datensatzes zumindest einige Regelmäßigkeiten festgestellt werden könnten, eine solche Analyse liegt jedoch außerhalb der Möglichkeiten dieses Projektes.

\clearpage
