% Chapter 5

\chapter{Whitespace} % Main chapter title

\label{Whitespace} % For referencing the chapter elsewhere, use \ref{Chapter1}

\lhead{\chaptername{} \thechapter{} - \emph{Whitespace}} % This is for the header on each page - perhaps a shortened title

%----------------------------------------------------------------------------------------

\section{Herleitung}

Anders als noch bei der Typographie oder beim Layout bewegen wir uns im Bereich der Struktur sehr weit von generell anwendbaren Regeln weg (auch wenn diese schon bei den genannten vorhergehenden Gebieten nicht absolut zu finden sind). Die Fragen, die sich in diesem Kapitel stellen sind also
\begin{enumerate}
 \item Gibt es überhaupt mathematische Grundlagen, die hier genutzt werden können?
 \item Wie werden die Inhalte so vermittelt, dass sie zum Einen universal anwendbar, zum Anderen aber auch so konkret wie möglich sind?
\end{enumerate}

%----------------------------------------------------------------------------------------

\section{Whitespace}

Das wichtigste Konzept in diesem Zusammenhang ist das des Whitespaces. Weiterhin solle hier auch des Gesetz der Nähe mit eingebracht werden. Zunächst sollte definiert werden, was Whitespace ist und warum man mit Whitespace arbeiten sollte.
Hier ist es schwer, absolute mathematische Regeln zu finden, die man in jedem Fall anwenden kann. Jedoch sollten sich restriktionen finden lassen, an denen sich der Nutzer später orientieren kann.
Die ersten Einsätze von Whitespace haben wir bereits bei der Typographier erlebt. generell macht ein Vorgehen "von innen nach außen" hier Sinn. So sollten zunächste Abstände von einzelnen Elementen festgelegt werden. Eine Menge dieser Elemente kann dann später zu einer Gruppe zusammengefasst werden (hier kommt das Gesetz der Nähe ins Spiel) und auch dieser Gruppe können wieder Abstände zugewiesen werden. So lässt sich die Struktur der gesamten Seite zumindest in einem gewissen festgelegtem Rahmen bestimmen.
Auch hier werden am Ende keine Meisterwerke heraus kommen, aber zumindest eine gewisse Grundqualität in der Struktur kann garantiert werden. Besonderes Augenmerk sollte hier auch darauf gelget werden, dass die Abstände konsistent sind.

%----------------------------------------------------------------------------------------

\section{Was ist Whitespace?}

Whitespace (oder auch "negative space") ist leerer Raum. Alex J. White beschreibt ihn wie folgt:

\begin{quote}
The single most overlooked element in visual design is emptiness. The lack of attention it receives explains the abundance of ugly and unread design. [...]
Design elements are always viewed in relation to their surroundings. Emptiness in two-dimensional design is called white space and lies behind the type and imagery. But it is more than just the background of a design, for if a design's background alone were properly constructed, the overall design would immediately double in clarity and usefulness. Thus, when it is used intriguingly, white space becomes foreground. The emptiness becomes a positive shape and the positive and negative areas become intricately linked.
\end{quote}

Whitespace ist also kein Raum, den es noch zu füllen gilt sondern bewusst leer gelassener Raum, dessen Aufgabe es ist, Elemente in ihrer Wichtigkeit zu unterstützen. Eine Faustregel ist dabei, dass Elemente, um die herum mehr Whitespace liegt wichtiger sind (vgl. Überschriften in der Typographie). Generell wirken Gestaltungen mit mehr Whitespace auch wertiger.
Hauptaugenmerk in diesem Projekt sollte aber darauf liegen, den Whitespace auf ein mindestmaß zu bringen um eine angenehme visuelle Gestaltung zu erhalten.

\clearpage
