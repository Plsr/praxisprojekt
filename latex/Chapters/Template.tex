\renewcommand{\chaptertitle}{App-Generator}

\chapter{\chaptertitle} % Main chapter title

\label{AppGenerator} % For referencing the chapter elsewhere, use \ref{Aufgabenstellung} 

%----------------------------------------------------------------------------------------

Um den Einstieg in die App-Entwicklung für Firefox-Apps bestmöglichst zu erleichtern, hatte sich das Entwicklungsteam entschieden einen sogenannten Yeoman Generator zu entwickeln, welcher ein Grundgerüst für eine Firefox-App erstellt (auch Scaffolding genannt).
Darunter zählt das Generieren der Ordnerstruktur innerhalb des Ordners „app“, welche bereits in Abschnitt \ref{ordnerstruktur} beschrieben wurde, das Erstellen einer einfachen bower.json Datei, sowie das Herunterladen der darin beschriebenen Abhängigkeiten mit bower in den Ornder „app/bower\_components“ das Erstellen einer package.json Datei, sowie das Herunterladen der darin beschriebenen Abhängigkeiten mit npm in den Ordner „node\_modules“ und dem Erstellen der in Abschnitt \ref{grunt} beschriebenen Gruntfile.js-Datei.
Außerdem wird nach der eigentlichen Generierung dieses Grundgerüsts, bereits der „Grunt Default-Job“ (siehe Abschnitt \ref{grunt}) ausgeführt, was zum Erstellen des Ordners „firefox\_debug“ führt.

\section{Yeoman}

Yeoman ist ein "{}SCAFFOLDING TOOL FOR MODERN WEBAPPS"{}\footnote{nach \cite{yeoman}}, also ein Werkzeug zum Erstellen von Grundgerüsten moderner Web-Applikationen.
Yeoman bietet zahlreiche Generatoren an, welche sehr einfach zu installieren und zu nutzen sind.
Um einen eigenen Yeoman Generator zu entwickeln, kann man sich eine Anleitung auf \url{http://yeoman.io/authoring/} anschauen.

Um Yeoman nutzen zu können müssen dies, als auch weitere abhängige Tools installiert werden.
Dies kann mit dem Befehl \texttt{npm install -g yo bower grunt-cli gulp} bewerkstelligt werden.

Mit dem Befehl \texttt{yo <generatorname>} kann nun ein bestimmter Generator eingesetzt werden, welcher jedoch wiederrum zuerst über npm installiert werden muss.

\section{FoxAngJS}

Der vom Entwicklungsteam erstellte Yeoman Generator nennt sich \textbf{FoxAngJS}.
Der Name wurde aus den Begriffen „Firefox OS“ und „AngularJS“ gebildet.

Möchte man diesen Generator nutzen, muss dieser, wie zuvor beschrieben, über npm heruntergeladen und installiert werden.
Dazu führt man den Befehl \texttt{npm install -g generator-foxangjs} aus.
Ist der Generator heruntergeladen und installiert kann dieser mit \texttt{yo foxangjs} genutzt werden.

\subsection{Ablauf des Generierungsprozesses}

Hier soll nun der Ablauf des Generierungsprozesses Schritt für Schritt beschrieben werden.

In Abbildung \ref{fig:foxangjs1} ist dargestellt, wie der Generator gestartet wird.
Zuerst wird ein neuer Ordner erstellt. 
In diesem Fall wurde dieser „TestApp“ genannt.
Als Nächstes wird in diesen Ordner gewechselt.
Darauffolgend wird mit Hilfe des Befehls \texttt{yo foxangjs} der Generator gestartet.

\begin{figure}[!ht]
  \centering
    \includegraphics[width=0.8\textwidth,keepaspectratio]{images/foxangjs_01.PNG}\\
  \caption{Schritt 1: Starten des Generators und Auswählen des App-Namens}
  \label{fig:foxangjs1}
\end{figure}

Der Generator stellt einem nun mehrere Fragen.
Die erste Frage zielt darauf hin den App-Namen zu ermitteln und in die Dateien, welche im Nachhinein erstellt werden, einzufügen.
Hinter der Frage steht in Klammern ein Standardwert, dieser ist in diesem Falle der Name des aktuellen Ordners.
Bestätigt man an dieser Stelle die Frage so wird „TestApp“ als App-Name verwendet.
Man kann jedoch einen anderen Namen eingeben.

Die zweite Frage (dargestellt in Abbildung \ref{fig:foxangjs2}) zielt darauf hin, der Applikation eine Versionsnummer zu geben.
Die Versionsnummer wird mit dem Standardwert 0.1.0 angegeben, kann jedoch auch hier geändert werden. \\

\begin{figure}[!ht]
  \centering
    \includegraphics[width=0.8\textwidth,keepaspectratio]{images/foxangjs_02.PNG}\\
  \caption{Schritt 2: Vergeben einer Versionsnummer}
  \label{fig:foxangjs2}
\end{figure}

Die dritte und letze Frage möchte den Entwicklernamen für die manifest.webapp-Datei ermitteln, welche genutzt wird um eine App im Firefox Marketplace zu veröffentlichen.
Da für den Entwicklernamen nur eine Person eingetragen werden kann (oder auch eine Organisation), wird wie in Abbildung \ref{fig:foxangjs3} die Person erfragt.
Als Standardwert wird der Name „John Doe“ angeboten und hier in der Abbildung durch „Someone Else“ ersetzt.

\begin{figure}[!ht]
  \centering
    \includegraphics[width=0.8\textwidth,keepaspectratio]{images/foxangjs_03.PNG}\\
  \caption{Schritt 3: Eingabe Entwicklername}
  \label{fig:foxangjs3}
\end{figure}

Nachdem nun der Name bestätigt wurde, beginnt die eigentliche Generierung der App.
Zuerst werden, wie in Abblidung \ref{fig:foxangjs4} dargestellt, die Ordnerstruktur erstellt und Dateien generiert. \\

\begin{figure}[!ht]
  \centering
    \includegraphics[width=0.8\textwidth,keepaspectratio]{images/foxangjs_04.PNG}\\
  \caption{Schritt 4: Generieren der Dateien und starten von npm install}
  \label{fig:foxangjs4}
\end{figure}

Daraufhin werden, wie in Abbildung \ref{fig:foxangjs4} bereits angedeutet, nacheinander die Befehle \texttt{npm install} und \texttt{bower install} ausgeführt.
Dies kann einige Minuten in Anspruch nehmen.
Es kann hier eventuell zu Fehlern kommen.
Sollten Fehler auftretten können die beiden Befehle selbst eingegeben werden.

Abbildung \ref{fig:foxangjs5} stellt dar wie die verschiedenen npm-Pakete installiert werden. \\

\begin{figure}[!ht]
  \centering
    \includegraphics[width=0.8\textwidth,keepaspectratio]{images/foxangjs_05.PNG}\\
  \caption{Schritt 5: Ausführen von npm install}
  \label{fig:foxangjs5}
\end{figure}

Abbildung \ref{fig:foxangjs6} stellt dar wie die bower-Bibliotheken installiert werden. \\

\begin{figure}[!ht]
  \centering
    \includegraphics[width=0.8\textwidth,keepaspectratio]{images/foxangjs_06.PNG}\\
  \caption{Schritt 6: Ausführen von bower install}
  \label{fig:foxangjs6}
\end{figure}

Sind die beiden vorherrigen Schritte erfolgreich durchgeführt worden, so wird automatisch der „Grunt Default-Job“ gestartet, welcher den firefox\_debug-Ordner erstellt und die App für die Firefox Web-IDE zusammenbaut.
Ist jedoch zuvor einer der Schritte fehlerhaft gewesen und wurden die beiden Befehle selbst durchgeführt, so muss auch der Befehl \texttt{grunt} manuell eingegeben und gestartet werden.
Das Ergebnis ist jedoch das selbe, siehe Abbildung \ref{fig:foxangjs7}.

\begin{figure}[!ht]
  \centering
    \includegraphics[width=0.8\textwidth,keepaspectratio]{images/foxangjs_07.PNG}\\
  \caption{Schritt 7: Fertige Installation, starten des Grunt Default-Jobs}
  \label{fig:foxangjs7}
\end{figure}

\subsection{Die App in der Firefox Web-IDE}

Wurde nun Schritt 7 durchgeführt kann mittels der Firefox Web-IDE der firefox\_debug-Ordner, welcher durch die Ausführung von Grunt generiert wurde, ausgewählt werden.
Wie dies in der Firefox Web-IDE aussieht wird in der Abbildung \ref{fig:foxangjs8} dargestellt. \\

\begin{figure}[!ht]
  \centering
    \includegraphics[width=0.8\textwidth,keepaspectratio]{images/foxangjs_08.PNG}\\
  \caption{firefox\_debug-Ordner in der Firefox Web-IDE}
  \label{fig:foxangjs8}
\end{figure}

Wird die App nun im Simulator gestartet kann man sehen wie sich die App auf einem Gerät verhalten würde (siehe Abbildung \ref{fig:foxangjs9}). \\

\begin{figure}[!ht]
  \centering
    \includegraphics[width=0.4\textwidth,keepaspectratio]{images/foxangjs_09.PNG}\\
  \caption{Die generierte App im Firefox OS Simulator (Version 2.2)}
  \label{fig:foxangjs9}
\end{figure}

An dieser Stelle soll noch einmal darauf hingewiesen werden, dass der firefox\_debug-Ordner nicht zum Entwickeln bereit gestellt wird, sondern zum statischen Testen der Applikation im Simulator oder auf einem Endgerät.
Die eigentliche Entwicklung findet im app-Ordner statt.
Durch die Nutzung von grunt-watch werden somit jegliche Änderungen an den Quelldateien festgestellt und der firefox\_debug-Ordner aktuallisiert.
Der Simulator muss somit nur neugestartet werden, um die Änderungen sehen zu können.

Möchte man seine später fertig-entwickelte App minifizieren, so befolgt man die Schritte in Abschnitt \ref{grunt}, in dem man den „Grunt Build-Job“ ausführt.

\clearpage