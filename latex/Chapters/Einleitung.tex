% Chapter 1

%\footnote{vgl. \cite{studie}}

\newcommand{\chaptertitle}{Einleitung}

\chapter{\chaptertitle} % Main chapter title

\label{Einleitung} % For referencing the chapter elsewhere, use \ref{Chapter1}

\lhead{\chaptername{} \thechapter{} - \emph{\chaptertitle}} % This is for the header on each page - perhaps a shortened title

%----------------------------------------------------------------------------------------

\section{Einleitung}
Die nachfolgende Dokumentation beschäftigt sich mit der Konzeption eines Tools, das dem Nutzer dabei helfen soll, grundlegende Gestaltungsregeln zu beachten.
Prägend für die Idee des Projektes ist dabei der Aufbau des Studiengangs “Medieninformatik” an der technischen Hochschule Köln. Dieser besteht zu großen Teilen aus Informatikmodulen, beinhaltet jedoch auch Module, die sich beispielsweise mit den Grundlagen der visuellen Kommunikation beschäftigen. Diese Grundlagen finden später in vielen studentischen Projekten keine Anwendung oder zeigen deutliche Mängel.

Das konzipierte Tool soll die nötigen Inhalte interaktiv vermitteln, den Nutzer also bewusst Fehler machen lassen und ihn auf diese Fahler aufmerksam machen. Weiterhin soll das Tool erläutern, warum die vom Nutzer gewählte Lösung nicht optimal ist.


%----------------------------------------------------------------------------------------

\section{Relevanz}
Zunächst soll hier jedoch die Relevanz eines solchen Tools diskutiert werden. In vielen der Module, die nicht explizit den Bereich Design zum Thema haben fließt die Gestaltung nicht mit in die Gesamtnote ein. Es stellt sich also die Frage, warum Studenten Zeit in die Gestaltung eines Artefaktes investieren sollten, wenn dadurch kein expliziter Mehrwert entsteht.

Zur Beantwortung dieser Frage sei zunächst das Konzept des *confirmation bias* (zu deutsch etwa “Voreingenommenheit”) erläutert. Raymond S Nickerson beschreibt dieses wie folgt:

\begin{quote}
“Confirmation bias, as the term is typically used in the psychological literature, connotes the seeking or interpreting of evidence in ways that are partial to existing beliefs, expectations, or a hypothesis in hand.“
\end{quote}
__[Raymond S Nickerson. Confirmation bias: A ubiquitous phenomenon in many guises. Review of general psychology, 2(2):175, 1998.]__

Menschen verwenden Informationen also so, dass sie als Argumente für ihre bisherige Meinung genutzt werden können und entwerten Argumente, die gegen ihre Meinung sprechen. Demnach lassen sich Menschen, sobald sie sich eine Meinung gebildet haben, nur schwer vom Gegenteil überzeugen.
Haben Menschen einen positiven Eindruck von etwas, so ist es sogar wahrscheinlich, dass sie später über Fehler hinwegsehen .__[Arnold Campbell and Susan Pisterman. A fitting approach to interactive service design: The importance of emotional needs. Design Management Journal (Former Series), 7(4):10–14, 1996.]__

Eine weiterer wichtiger Aspekt scheint also die Bildung des ersten Eindrucks über eine Sache zu sein. Lindgaard, Fernandes, Dudek und Brown haben festgestellt, das Menschen ein Zeitraum von 50ms ausreicht, um sich eine Meinung über eine Webseite zu bilden. __[Gitte Lindgaard, Gary Fernandes, Cathy Dudek, and Judith Brown. Attention web designers: You have 50 milliseconds to make a good first impression! Behaviour & information technology, 25(2):115–126, 2006.]__

Setzt man diese beiden Elemente in Abhängigkeit wird auch die Relevanz dieses Projektes deutlich: Menschen bilden sich sehr schnell ein Urteil über die Gestaltung eines Artefaktes und dieses lässt sich nur schwer wieder ändern.
Die oben dargelegten Quellen lassen den Schluß zu, dass dieser schlechte Eindruck auch auf andere, vielleicht Bewertungsrelevante Kriterien übertragen wird.


%----------------------------------------------------------------------------------------

\section{Zielsetzung}
Ziel des Tools ist es, in Artefakten eine gewisse gestalterische Grundqualität auf visueller Ebene zu sichern. Dabei sollte es möglichst unabhängig vom Artefakt sein, sich also sowohl auf ein Textdokument als auch eine Android-App anwenden lassen. Mit dieser Interdisziplinarität geht natürlich ein Verlust in Details einher, der bewusst in Kauf genommen wird.
Mit der Ambition, nicht abhängig vom Artefakt zu sein muss das Tools selbst eine generelle Zugänglichkeit bieten und darf sich nicht hinter Beschränkung einer bestimmten Plattform verbergen. Hier liegt die Umsetzung als WebApp nahe. Auch potenzielle Ergebnisse des Tools müssen dabei Problemlos auf verschiedene Anwendungsgebiete übertragbar sein.
Die Zielgruppe sind hierbei explizit Studenten des Studienganges Medieninformatik an der Technischen Hochschule Köln, generell aber Nutzer, die aus verschiedenen Gründen kein oder nur wenig Wissen und Anwendungserfahrung mit visueller Gestaltung haben.

%----------------------------------------------------------------------------------------

\section{Forschungsfrage}
Die Forschungsfrage spaltet sich in zwei Teile: Zum Einen müssen relevante Themengebiete mitsamt ihrer Regeln für ein solches Tool erarbeitet werden, zum anderen muss eine Form gefunden werden, in der die erarbeiteten Themen und Regeln dem Nutzer nahegelegt werden sollen.
Die ausformulierte Forschungsfrage lautet somit:

“Ermittlung relevanter Themengebiete und Konzeption derer interaktiven Darstellung für die Entwicklung eines Tools zum erstellen von Gestaltungslösungen im Hochschulkontext”



%----------------------------------------------------------------------------------------

\section{Vorgehensweise}
Als erster Schritt mussten die relevanten Themengebiete definiert werden. Hierfür musste eine Methode zur Ermittlung gefunden werden. Eine genauere Dokumentation dieses Schrittes findet sich in Kapitel 2.

Darauf folgend mussten in den definierten Themengebieten Regeln gefunden werden, die nach Möglichkeit auch programmatisch abgebildet werden können.
In diesem Schritt fand sich auch die größte Herausforderung des Projektes. Beim finden von Gestaltungslösungen gibt es oft keinen richtigen oder falschen weg und viele Entscheidungen werden auf einer subjektiven Ebene getroffen. Daher soll das Tool dem Nutzer eher helfen, Grundlagen zu schaffen, als konkrete Lösungen zu finden.

Für jeden dieser Bereiche wurden die Ergebnisse in Mockups oder funktionalen Prototypen festgehalten.


%----------------------------------------------------------------------------------------

\clearpage
