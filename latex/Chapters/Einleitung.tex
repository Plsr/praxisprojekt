% Chapter 1

%\footnote{vgl. \cite{studie}}

\newcommand{\chaptertitle}{Einleitung}

\chapter{\chaptertitle} % Main chapter title

\label{Einleitung} % For referencing the chapter elsewhere, use \ref{Chapter1} 

\lhead{\chaptername{} \thechapter{} - \emph{\chaptertitle}} % This is for the header on each page - perhaps a shortened title

%----------------------------------------------------------------------------------------

\section{Einleitung}
Das Praxisprojekt beschäftigt sich mit der Ermittlung relevanter Themengebiete und dem 
Finden einer Methode zur Vermittlung wichtiger Konzepte für die Entwicklung eines Tools 
zur Unterstützung beim erstellen von Gestaltungslösungen im Hochschulkontext. 
Im folgenden soll zunächst das Ziel des Tools genauer spezifiziert werden und eine wissenschaftliche Grundlage für die existenz des Tool hergeleitet werden, sowie die Kerngebiete dargelegt werden.

%----------------------------------------------------------------------------------------

\section{Zielsetzung}
Ziel des Tools ist es, in Artefakten eine gewisse gestalterische Grundqualität auf visueller Ebene zu sichern. Dabei sollte es möglichst unabhängig vom Artefakt sein, sich also sowohl auf ein Textdokument als auch eine Android-App anwenden lassen. Mit dieser Interdisziplinarität geht natürlich ein Verlust in Details einher, der bewusst in Kauf genommen wird.
Mit der Ambition, nicht abhängig vom Artefakt zu sein muss das Tools selbst eine generelle Zugänglichkeit bieten und darf sich nicht hinter Beschränkung einer bestimmten Plattform verbergen. Hier liegt die Umsetzung als WebApp nahe. Auch potenzielle Ergebnisse des Tools müssen dabei Problemlos auf verschiedene Anwendungsgebiete übertragbar sein.
Die Zielgruppe sind hierbei explizit Studenten des Studienganges Medieninformatik an der Technischen Hochschule Köln, generell aber Nutzer, die aus verschiedenen Gründen kein oder nur wenig Wissen und Anwendungserfahrung mit visueller Gestaltung haben.

%----------------------------------------------------------------------------------------

\section{Wissenschaftlich belegte Relevanz}
Die Wissenschaftliche Grundlage für die Existenz des Tools liefern zwei Konzepte und eine Studie aus dem Bereich der Psychologie und der Wahrnehmung.
Als erstes sei hier die Studie von Lindgaard, Fernandes, Dudek und Brown genannt, die untersucht wie schnell sich Menschen ein Urteil über die Gestaltungsqualität einer Website bilden, also für sich selbst feststellen, ob eine Seite gut oder schlecht gestaltet ist.
\begin{quote}
“The above data suggest that a reliable decision can be made in 50 ms, which supports the contention that judgements of visual appeal could represent a mere exposure effect.“ 
\end{quote}
Hieraus wird klar, dass ein Nutzer sich keine Zeit nimmt, die Gestaltung einer Seite (oder auch eines anderen Artefaktes) zu analysieren, bevor er zu einem Entschluss über dessen Qualität kommt. Eine gute und saubere Gestaltung ist also elementar wichtig.

Das erste Konzept, das hier von Bedeutung ist, ist das des Confirmatory Bias. Dieses Konzept ist viel erforscht worden und besagt, dass Menschen Informationen so verwenden, dass sie als Argumente für ihre eigene Meinung genutzt werden ko ̈nnen und andere Argumente entwerten. Demnach lassen sich Menschen, sobald sie sich eine Meinung gebildet haben, nur schwer vom Gegenteil überzeugen.
\begin{quote}
“Confirmation bias, as the term is typically used in the psychological literature, connotes the seeking or interpreting of evidence in ways that are partial to existing beliefs, expectations, or a hypothesis in hand.“
\end{quote}


%----------------------------------------------------------------------------------------

\section{Aktueller Stand in Markt und Forschung}

%----------------------------------------------------------------------------------------

\clearpage