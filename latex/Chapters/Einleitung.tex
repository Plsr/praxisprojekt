% Chapter 1

\chapter{Einleitung} % Main chapter title

\label{Einleitung} % For referencing the chapter elsewhere, use \ref{Chapter1}

\lhead{\chaptername{} \thechapter{} - \emph{Einleitung}} % This is for the header on each page - perhaps a shortened title

%----------------------------------------------------------------------------------------
Die nachfolgende Dokumentation hält Arbeitsschritte und Entscheidungen fest, die bei der  Konzeption eines Tools getroffen wurden, das dem Nutzer dabei helfen soll, grundlegende Gestaltungsregeln im Hochschulkontext zu beachten.
Prägend für die Idee des Projektes ist dabei der Aufbau des Studiengangs “Medieninformatik” an der Technischen Hochschule Köln. Dieser besteht zu großen Teilen aus Informatikmodulen, beinhaltet jedoch auch Module, die sich beispielsweise mit den Grundlagen der visuellen Kommunikation beschäftigen. Die in diesen Modulen vermittelten Grundlagen finden jedoch in späteren studentischen Projekten häufig keine Anwendung mehr.
Einer der ersten Arbeitsschritte war es dabei, einen Beweis für die Relevanz eines solchen Tools zu erbringen. Dieses erste Kapitel beschäftig sich mit der Erbringung dieses Beweises und der allgemeinen Zielsetzung des Projektes.

%----------------------------------------------------------------------------------------

\section{Relevanz}
In vielen der Module, die nicht explizit im Bereich Design und Gestaltung stattfinden, fließen diese Disziplinen auch nicht mit in die Bewertung ein. Es stellt sich also die Frage, warum Studenten Zeit und Aufwand in die Gestaltung eines Artefaktes investieren sollten, wenn dadurch offenbar kein expliziter Mehrwert entsteht.

Zur Beantwortung dieser Frage sei zunächst das Konzept des \textit{confirmation bias} (zu deutsch etwa “Voreingenommenheit”) erläutert. Raymond S Nickerson beschreibt dieses wie folgt:

\begin{quote}
“Confirmation bias, as the term is typically used in the psychological literature, connotes the seeking or interpreting of evidence in ways that are partial to existing beliefs, expectations, or a hypothesis in hand.“ \cite{nickerson1998confirmation}
\end{quote}

Menschen verwenden Informationen also so, dass sie als Argumente für ihre bisherige Meinung genutzt werden können und entwerten Argumente, die gegen ihre Meinung sprechen. Demnach lassen sich Menschen, sobald sie sich eine Meinung gebildet haben, nur schwer vom Gegenteil überzeugen.
Haben Menschen einen positiven Eindruck von etwas, so ist es sogar wahrscheinlich, dass sie später über Fehler hinwegsehen \cite{campbell1996fitting}

Ein weiterer wichtiger Aspekt scheint also die Bildung des ersten Eindrucks über eine Sache zu sein. Lindgaard, Fernandes, Dudek und Brown haben festgestellt, dass Menschen ein Zeitraum von 50ms ausreicht, um sich eine Meinung über eine Webseite zu bilden. \cite{lindgaard2006attention}

Setzt man diese beiden Elemente in Abhängigkeit, wird auch die Relevanz dieses Projektes deutlich: Menschen bilden sehr schnell ein Urteil über die Gestaltung eines Artefaktes und dieses lässt sich nur schwer wieder ändern. Weiterhin wird dieser schlechte erste Eindruck auf andere Bereiche des Artefaktes übertragen und wirkt sich somit unter Umständen auch auf die Gesamtbewertung eines Artefaktes aus.

%----------------------------------------------------------------------------------------

\section{Zielsetzung}
Das Ziel des Tools ist, dem Nutzer zu helfen, in Artefakten eine gewisse gestalterische Grundqualität zu sichern. Dabei sollte es möglichst unabhängig vom Artefakt sein, sich also sowohl auf ein Textdokument als auch eine Android-App anwenden lassen (mit dieser Interdisziplinarität geht natürlich ein Verlust bei der Fokussierung auf Details einher, der bewusst in Kauf genommen wird).
Mit der Ambition, nicht abhängig vom Artefakt zu sein, muss das Tools selbst eine generelle Zugänglichkeit bieten und sollte nicht den Beschränkungen einer bestimmten Plattform unterliegen. Hier liegt die Umsetzung als WebApp nahe. Auch potenzielle Ergebnisse des Tools müssen dabei Problemlos auf verschiedene Anwendungsgebiete übertragbar sein.
Die Zielgruppe sind hierbei zunächst Studenten des Studienganges Medieninformatik an der Technischen Hochschule Köln, diese kann aber auch auf Nutzer ausgeweitet werden, die aus verschiedenen Gründen keine oder nur wenig Anwendungserfahrung im Bereich der visuellen Gestaltung haben.
Das konzipierte Tool soll die nötigen Inhalte interaktiv vermitteln, den Nutzer also bewusst Fehler machen lassen und ihn auf diese Fehler aufmerksam machen. Weiterhin soll das Tool erläutern, warum die vom Nutzer gewählte Lösung nicht optimal ist.

Eine komplette Umsetzung dieser Projektidee im Rahmen des Praxisprojektes würde dessen Umfang sprengen, daher soll sich das Praxisprojekt zunächst mit der Konzipierung eines solchen Tools befassen. Konkret bedeutet dies für das Projekt, dass zunächst relevante Themengebiete für ein solches Tool gefunden werden müssen. Darauf aufbauend müssen in diesen Themengebieten Regeln erarbeitet werden, auf dessen Grundlage ein solches Tool umgesetzt werden kann. Weiterhin sollen mögliche Umsetzungen als \textit{Wireframes} oder kleine \textit{Proof of Concepts} festgehalten werden.

Die ausformulierte Forschungsfrage für das Projekt lautet somit:

\textit{“Ermittlung relevanter Themengebiete und Konzeption derer interaktiven Darstellung für die Entwicklung eines Tools zum Erstellen von Gestaltungslösungen im Hochschulkontext”}


%----------------------------------------------------------------------------------------

\clearpage
