% Chapter 1

\chapter{Einleitung} % Main chapter title

\label{Einleitung} % For referencing the chapter elsewhere, use \ref{Chapter1}

\lhead{\chaptername{} \thechapter{} - \emph{Einleitung}} % This is for the header on each page - perhaps a shortened title

%----------------------------------------------------------------------------------------
Die nachfolgende Dokumentation hält Arbeitsschritte und Entscheidungen fest, die bei der  Konzeption eines Tools getroffen wurden, das dem Nutzer dabei helfen soll grundlegende Gestaltungsregeln im Hochschulkontext zu beachten.
Prägend für die Idee des Projektes ist dabei der Aufbau des Studiengangs “Medieninformatik” an der Technischen Hochschule Köln. Dieser besteht zu großen Teilen aus Informatikmodulen, beinhaltet jedoch auch Module, die sich beispielsweise mit den Grundlagen der visuellen Kommunikation beschäftigen. Die in diesen Modulen vermittelten Grundlagen finden jedoch in späteren studentischen Projekten häufig keine Anwendung mehr.
Einer der ersten Arbeitsschritte war es dabei, einen Beweis für die Relevanz eines solchen Tools zu erbringen. Dieses erste Kapitel beschäftig sich mit der Erbringung dieses Beweises und der allgemeinen Zielsetzung des Projektes.

%----------------------------------------------------------------------------------------

\section{Relevanz}
In vielen der Module, die nicht explizit im Bereich Design und Gestaltung stattfinden fließen diese Disziplinen auch nicht mit in die Bewertung ein. Es stellt sich also die Frage, warum Studenten Zeit und Aufwand in die Gestaltung eines Artefaktes investieren sollten, wenn dadurch offenbar kein expliziter Mehrwert entsteht.

Zur Beantwortung dieser Frage sei zunächst das Konzept des \textit{confirmation bias} (zu deutsch etwa “Voreingenommenheit”) erläutert. Raymond S Nickerson beschreibt dieses wie folgt:

\begin{quote}
“Confirmation bias, as the term is typically used in the psychological literature, connotes the seeking or interpreting of evidence in ways that are partial to existing beliefs, expectations, or a hypothesis in hand.“ \cite{nickerson1998confirmation}
\end{quote}

Menschen verwenden Informationen also so, dass sie als Argumente für ihre bisherige Meinung genutzt werden können und entwerten Argumente, die gegen ihre Meinung sprechen. Demnach lassen sich Menschen, sobald sie sich eine Meinung gebildet haben, nur schwer vom Gegenteil überzeugen.
Haben Menschen einen positiven Eindruck von etwas, so ist es sogar wahrscheinlich, dass sie später über Fehler hinwegsehen \cite{campbell1996fitting}

Eine weiterer wichtiger Aspekt scheint also die Bildung des ersten Eindrucks über eine Sache zu sein. Lindgaard, Fernandes, Dudek und Brown haben festgestellt, das Menschen ein Zeitraum von 50ms ausreicht, um sich eine Meinung über eine Webseite zu bilden. \cite{lindgaard2006attention}

Setzt man diese beiden Elemente in Abhängigkeit wird auch die Relevanz dieses Projektes deutlich: Menschen bilden sehr schnell ein Urteil über die Gestaltung eines Artefaktes und dieses lässt sich nur schwer wieder ändern. Weiterhin wird dieser schlechte erste Eindruck auf andere Bereiche des Artefaktes übertragen und wirkt sich somit unter Umständen auch auf die Gesamtbewertung eines Artefaktes aus.

%----------------------------------------------------------------------------------------

\section{Zielsetzung}
Ziel des Tools ist es dem Nutzer zu helfen,  in Artefakten eine gewisse gestalterische Grundqualität auf visueller Ebene zu sichern. Dabei sollte es möglichst unabhängig vom Artefakt sein, sich also sowohl auf ein Textdokument als auch eine Android-App anwenden lassen (mit dieser Interdisziplinarität geht natürlich ein Verlust bei der Fokussierung auf Details einher, der bewusst in Kauf genommen wird).
Mit der Ambition, nicht abhängig vom Artefakt zu sein muss das Tools selbst eine generelle Zugänglichkeit bieten und sollte nicht den Beschränkung einer bestimmten Plattform unterliegen. Hier liegt die Umsetzung als WebApp nahe. Auch potenzielle Ergebnisse des Tools müssen dabei Problemlos auf verschiedene Anwendungsgebiete übertragbar sein.
Die Zielgruppe sind hierbei zunächst Studenten des Studienganges Medieninformatik an der Technischen Hochschule Köln, diese kann aber auch auf Nutzer ausgeweitet werden, die aus verschiedenen Gründen kein oder nur wenig Wissen und Anwendungserfahrung mit visueller Gestaltung haben.
Das konzipierte Tool soll die nötigen Inhalte interaktiv vermitteln, den Nutzer also bewusst Fehler machen lassen und ihn auf diese Fahler aufmerksam machen. Weiterhin soll das Tool erläutern, warum die vom Nutzer gewählte Lösung nicht optimal ist.

%----------------------------------------------------------------------------------------

\section{Forschungsfrage}
Die Forschungsfrage spaltet sich in zwei Teile: Zum Einen müssen relevante Themengebiete mitsamt ihrer Regeln für ein solches Tool erarbeitet werden, zum anderen muss eine Form gefunden werden, in der die erarbeiteten Themen und Regeln dem Nutzer nahegelegt werden sollen.
Die ausformulierte Forschungsfrage lautet somit:

\textit{“Ermittlung relevanter Themengebiete und Konzeption derer interaktiven Darstellung für die Entwicklung eines Tools zum erstellen von Gestaltungslösungen im Hochschulkontext”}



%----------------------------------------------------------------------------------------

\section{Vorgehensweise}
Als erster Schritt mussten die relevanten Themengebiete definiert werden. Hierfür musste eine Methode zur Ermittlung gefunden werden. Eine genauere Dokumentation dieses Schrittes findet sich in Kapitel 2.

Darauf folgend mussten in den definierten Themengebieten Regeln gefunden werden, die nach Möglichkeit auch programmatisch abgebildet werden können.
In diesem Schritt fand sich auch die größte Herausforderung des Projektes. Beim finden von Gestaltungslösungen gibt es oft keinen richtigen oder falschen weg und viele Entscheidungen werden auf einer subjektiven Ebene getroffen. Daher soll das Tool dem Nutzer eher helfen, Grundlagen zu schaffen, als konkrete Lösungen zu finden.

Für jeden dieser Bereiche wurden die Ergebnisse in Mockups oder funktionalen Prototypen festgehalten.


%----------------------------------------------------------------------------------------

\clearpage
