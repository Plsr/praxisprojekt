% Chapter 5

\chapter{Fazit} % Main chapter title

\label{Fazit} % For referencing the chapter elsewhere, use \ref{Aufgabenstellung} 

\lhead{\chaptername{} \thechapter{} - \emph{Fazit}} % This is for the header on each page - perhaps a shortened title

Abschließend lässt sich die Zielsetzung als erreicht ansehen. \\
Es wurden relevante Themengebiete für die Konzeption eines Tools definiert. In diesen Themengebieten wurden Regeln festgehalten, die auch für eine spätere programmatische Umsetzung verwendet werden können. In machen Gebieten, wie der Typographie oder den Farben, lassen diese Regeln mehr Spielraum für Interaktivität zu, in Anderen, wie beispielsweise den Interaktiven Elementen und Bildern, weniger. Lediglich im Bereich Whitespace konnten keinerlei Regeln definiert werden.

Mögliche Ideen für eine Umsetzung wurden in Form von Wireframes festgehalten und im Bereich der Typographie konnte mit Hilfe eines \textit{Proof of Concepts} gezeigt werden, dass auch eine tatsächliche technische Umsetzung mit moderatem Aufwand durchaus möglich ist.

Während des Projektes konnte aber auch festgestellt werden, dass das Übertragen von eher subjektiven Regeln für die Gestaltung von Artefakten in eine objektive, maschinenlesbare Form nicht einfach ist. Viele Entscheidungen, die Designer jeden Tag treffen, kann ein Computer nicht nachvollziehen. Auch können getroffene Entscheidungen nicht einfach als richtige oder falsche Referenzwerte übergeben werden, da deren Richtigkeit stark situationsabhängig ist. \\
Hier stellt sich die Frage, ob der Bereich \textit{Machine Learning} eine sinnvolle Ergänzung für das Projekt bieten könnte, also ob ein Computer, ähnlich wie ein Mensch, ein \textit{Gefühl} für eine gute oder schlechte Gestaltung entwickeln kann, wenn seine Datengrundlage groß genug ist.

Weiterhin ist auch die Frage nach dem tatsächlichen Nutzen den Tools noch ungeklärt. Zwar wird deutlich, das ein solches Tool eine gewisse Relevanz hat und auch umgesetzt werden kann, es kann jedoch nicht Festgestellt werden, ob das Tool dem Nutzer einen tatsächlichen Mehrwert liefern würde. Hierfür wären Nutzertests mit Prototypen nötig. 


\clearpage