% Chapter 2

\chapter{Ermittlung relevanter Themengebiete} % Main chapter title

\label{Ermittlung} % For referencing the chapter elsewhere, use \ref{Chapter1}

\lhead{\chaptername{} \thechapter{} - \emph{Ermittlung relevanter Themengebiete}} % This is for the header on each page - perhaps a shortened title

%----------------------------------------------------------------------------------------

Für die Ermittlung der relevanten Themengebiet musste zunächst eine geeignete Methode gefunden werden. Hierfür kamen verschiedene Ansätze in Frage.

Ein möglicher Ansatz orientiert sich an einer von einer qualifizierten Person bereits definierten Liste von Inhalten, beispielsweise an einem Buch oder einem Studienverlaufsplan.
Dieser Ansatz gewährleistet durch die Fähigkeiten der Ersteller zum Einen und die große Auswahl von Materialien und damit hohe Vergleichbarkeit zum Anderen einen gewissen Grad von Fehlerfreiheit.
Die definierten Inhalte richten sich jedoch zumeist an Personen, deren Haupttätigkeit die visuelle Gestaltung ist. Bei Verwendung dieser Methode müssen also die Menge und Tiefe der Inhalte kritisch geprüft werden.

Ein weiterer Ansatz orientiert sich am Workflow der Benutzer und beschäftigt sich zunächst mehr mit der Frage nach dem “wann” als mit der nach dem “was”. Der Ansatz würde sich also daran orientieren, wann ein Benutzer eine bestimmte Tätigkeit ausführt und welche Art der Unterstützung an dieser Stelle vom Tool einzubringen ist.
Dieser Ansatz wurde schnell als ungeeignet verworfen, da mit ihm ein hoher Aufwand in der Recherche von Arbeitsabläufen der Nutzer einher geht.

Für die Ermittlung wurde schlußendlich ein Fehlerorientierter Ansatz verwendet. Bei diesem Ansatz wurden Artefakte auf die Fehler hin untersucht, die Nutzer häufig machen und die Inhalte des Tools daraufhin angepasst.
Die Wahl dieser Methode liegt gerade deshalb nahe, weil das Tool zunächst im Hochschulkontext Verwendung finden sollte. Durch das Mediawiki des Studienganges Medieninformatik war der Zugriff auf viele Materialien gewährleistet, die auf Fehler hin untersucht werden konnten. Weiterhin konnte das Tools mithilfe dieses Ansatzes so genau wie möglich auf die Themengebiete spezialisiert werden, in denen die Nutzer häufig Defizite aufzeigen.

%----------------------------------------------------------------------------------------

\section{Vorgehen}
Im erstern Schritt wurden, vorrangig im Mediawiki des Studienganges Medieninformatik, Materialien gesammelt. Um sowohl das Sommer- als auch das Wintersemester und somit alle angebotenen Module abzubilden wurden alle Dateien untersucht, die im Zeitraum des vergangenen Jahres hochgeladen wurden.
Da nicht jedes der in der Medieninformatik angebotenen Module das Mediawiki verwendet wurde versucht, auch außerhalb des Wikis Materialien zu finden. Hier gestaltete sich die Suche allerdings weitaus aufwendiger und weniger erfolgreich.
Insgesamt wurde ein Katalog von 448 Dateien erstellt, die zur Orientierung zunächst grob in die vier Kategorien \textit{Präsentation, Plakat, Interaktiv} und \textit{Text} unterteilt wurden.

Zur Gruppe \textit{Interaktiv} sei hier angemerkt, dass diese bewusst weit gefasst ist. Sie umfasst sowohl Android-Apps, als auch Websites mit Fokus auf Front- oder Backend.
Eine weitere Unterteilung wurde als nicht Sinnvoll erachtet. So hätten Plattformspezifische Fehler zwar besser gefunden werden können, jedoch soll sich das Tool eher mit gestalterischen Grundlage befassen und ein möglichst weiter Spektrum abdecken. (Weitere Erläuterungen zu den mobilen Plattformen finden sich im Abschnitt \ref{Mobile Plattformen} auf Seite \pageref{Mobile Plattformen})

Die genauen Zahlen der Artefakte in den jeweiligen Kategorien können Tabelle \ref{table:types} auf Seite \pageref{table:types} entnommen werden. Bei der Betrachtung der Werte fällt auf, dass nahezu 50\% der untersuchten Artefakte in die Kategorie “Text” fallen. Weiterhin fällt auf, dass Plakate mit etwa 3\% aller Artefakte kaum vorhanden waren.

\begin{table}[]
\centering
\begin{tabular}{|l|c|}
\hline
\textbf{Typ} & \multicolumn{1}{l|}{\textbf{Anzahl der Dateien}} \\ \hline
Präsentation & 96                                               \\ \hline
Plakat       & 16                                               \\ \hline
Interaktiv   & 113                                              \\ \hline
Text         & 223                                              \\ \hline
\end{tabular}
\caption{Auflistung der untersuchten Artefakte nach Typ}
\label{table:types}
\end{table}

Das gesammelte Material wurde im nächsten Schritt auf Fehler in der Gestaltung untersucht. Nachfolgend findet sich eine Liste der gefundenen Fehler (in absteigender Reihenfolge nach Häufigkeit ihres Auftretens):

\begin{itemize}
	\item \textbf{Fehler bim Einsatz von Farben:} Es wurden Farbkombinationen verwendet die flimmerten, visuell anstrengende Farben wurden als Hintergrundfarbe für Texte gewählt, Unharmonische Farbkombinationen, übermäßiger Einsatz von Farben.
	\item \textbf{Fehlendes raster:} Beim Ausrichten der Elemente wurde kein Raster verwendet, Elemente wirken dadurch wie zufällig platziert.
	\item \textbf{Zu wenig Weissraum:} In der Gestaltung wurde zu wenig Whitespace verwendet, die gesamte Gestaltung ist dadurch unübersichtlich und wirkt unruhig.
	\item \textbf{Visuell anstrengende Tabellen:} Tabellen verwendeten mehr Trennlinien als nötig, verwendeten zu viele/unpassende Farben, es war schwer die Informationen aus der Tabelle zu entnehmen, es wurden Tabellen verwendet wo keine nötig gewesen wären.
	\item \textbf{Hierarchie im Text nicht deutlich:} Die Unterscheide zwischen verschiedenen Elementen(z.B. Überschriften verschiedener Ordnungen) waren zu gering, es wurde keine klare Hierarchie deutlich.
	\item \textbf{Fehlerhafte Ausrichtung von Elementen:} Elemente wurden in sich unsauber ausgerichtet
	\item \textbf{Interaktive Elemente nicht deutlich:} Es wurde nicht deutlich, mit welchen Elementen der Benutzer interagieren kann und mit welchen nicht.
	\item Bilder gestreckt/gestaucht/verpixelt
	\item \textbf{Inkonsistente Größen:} Gleiche Elemente waren verscheiden groß
	\item \textbf{Elementgrößen unverhältnismäßig:} Einige Elemente sind im Vergleich zu anderen Elementen unverhältnismäßig zu klein/zu groß
	\item \textbf{Zu wenig Kontrast im Text:} Text war wegen des Kontrastes schwer lesbar
	\item \textbf{Boxes in Boxes:} Unnötiges Verwenden von Boxen
	\item Zeilenhöhe zu groß/klein
	\item \textbf{Fehler im Textsatz:} Vor allem: Lücken im Blocksatz
	\item Rechtschreibfehler
	\item Schlecht lesbare Schriftfamilie
\end{itemize}


Abschließend wurde jedem der Fehler eine Oberdisziplin zugeordnet. Auch im Hinblick auf die zur Verfügung stehende Projektzeit wurden darauf aufbauen die Themengebiete definiert, die das Tool abdecken soll:
\begin{itemize}
	\item Typographie
	\item Layout & Struktur
	\item Whitespace
	\item Farben
	\item Interaktive Elemente
	\item Bilder
\end{itemize}


%----------------------------------------------------------------------------------------

\section{Kritische Reflexion}
Auch wenn dieses Fehlerorientierte vorgehen für das Projekt sehr hilfreich ist bringt es einige Gefahren mit sich, die hier kurz diskutiert werden sollen.

So ist das Material von nur zwei Semestern keineswegs dazu geeignet, empirische Ergebnisse zu liefern. Die erarbeiteten Ergebnisse liefern lediglich einen Überblick über die Fehler, die in naher Vergangenheit vorherrschend waren. Inhalte von Modulen ändern sich häufig von Semester zu Semester, eine Untersuchung im nächsten Jahr würde also mit hoher Wahrscheinlichkeit andere Fehler zeigen.

Weiterhin wurde nicht für alle angebotenen Module auch Materialien gefunden. Es ist also durchaus möglich, dass hier Fehler überhaupt nicht entdeckt wurden.

Als Startpunkt für das Projekt ist die oben aufgeführte Liste jedoch durchaus geeignet.


%----------------------------------------------------------------------------------------

\section{Mobile Plattformen} \label{Mobile Plattformen}
Die Mobilen Plattformen, \textit{Android} und \textit{iOS}, die im Studiengang Medieninformatik verwendet werden besitzen eigene Design Guidelines, die für bestimmte Themengebiete schon Vorgaben und Richtlinien bereit stellen. So legen sowohl die Guidelines für Android als auch für iOS legen die Verwendung einer oder mehrerer bestimmter Schriftfamilien nahe.

Da sich dieses Tool nicht über die plattformspezifischen Richtlinien hinwegsetzen sollte, sollte von Anfang an die Plattform  auf der das Projekt des Nutzers laufen soll bekannt sein. Die Inhalte des Tools sollten sich dementsprechend anpassen.

%----------------------------------------------------------------------------------------

\clearpage
