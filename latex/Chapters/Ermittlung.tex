% Chapter 1

%\footnote{vgl. \cite{studie}}

\newcommand{\chaptertitle}{Ermittlung relevanter Themengebiete}

\chapter{\chaptertitle} % Main chapter title

\label{Ermittlung} % For referencing the chapter elsewhere, use \ref{Chapter1}

\lhead{\chaptername{} \thechapter{} - \emph{\chaptertitle}} % This is for the header on each page - perhaps a shortened title

%----------------------------------------------------------------------------------------

\section{Einleitung}
Für die Ermittlung der relevanten Themengebiet musste zunächst eine geeignete Methode gefunden werden, hierfür kamen verschiedene Ansätze in Frage.

Der simpelste Ansatz orientiert sich an den bereits definierten Inhalten eines Buches oder Studienganges, die von qualifizierten Leuten erstellt wurden, zu orientieren.
Dieser Ansatz gewährleistet durch die Fähigkeiten der Ersteller zum Einen und die große Auswahl und damit hohe Vergleichbarkeit zum anderen eine gewisse Fehlerfreiheit.
Die Inhalte richten sich jedoch an Personen, deren Haupttätigkeit die visuelle Gestaltung ist. Bei Verwendung dieser Methode müssen also die Menge und Tiefe der Inhalte kritisch geprüft werden.

Ein weiterer Ansatz orientiert sich am Wrokflow der Benutzer und beschäftigt sich eher mit der Frage nach dem “wann” als nach der mit dem “was”. Der Ansatz würde sich also daran orientieren, wann ein User eine bestimmte Tätigkeit ausführt und welche Bereiche der visuellen Gestaltung an dieser Stelle einzubringen sind.
Dieser Ansatz wurde schnell als ungeeignet verworfen, da mit ihm ein hoher Aufwand in der Recherche von Arbeitsabläufen der Nutzer einher geht.

Ein weiterer Ansatz war ein Fehlerorientierter Ansatz, für den ich mich schlußendlich auch entschied. Mit diesem Ansatz würden die Fehler Untersucht, die Nutzer häufig machen und die Inhalte des Tools daraufhin angepasst.
Die Wahl dieser Methode liegt gerade deshalb nahe, weil das Tool zunächst im Hochschulkontext Verwendung finden sollte. Durch das Mediawiki des Studienganges Medieninformatik war der Zugriff auf viele Materialien gewährleistet, die auf Fehler untersuch werden konnten.

%----------------------------------------------------------------------------------------

\section{Vorgehen}
Als erster Schritt wurden Materialien gesammelt. Der erste Anlaufpunkt hierfür war das Mediawiki des Studienganges Medieninformatik. Um zu gewährleisten, dass jedes Modul im Katalog vertreten war, wurden alle Dateien untersucht, die im letzten Jahr hochgeladen wurden.
Da nur ein Teil der Module das Mediawiki verwendet, wurde weiterhin auch außerhalb dessen nach Material gesucht, wovon sich das meiste auf GitHub wiederfand.
Insgesamt wurde so ein Katalog von 448 Dateien erstellt.

Diese wurden zur Orientierung zunächst grob in vier Kategorien unterteilt:

* Präsentation
* Plakat
* Interaktiv
* Text

Zur Gruppe *Interaktiv* sei hier angemerkt, dass diese bewusst weit gefasst ist. Sie umfasst sowohl Android-Apps, als auch Websites mit Fokus auf Front- oder Backend.
Eine weiter Unterteilung wurde an nicht als Sinnvoll erachtet, da so zwar Plattformspezifische Fehler besser hätten gefunden werden können, das Tool sich jedoch auf die Grundlagen Konzentrieren soll. (Weitere Erläuterungen zu den Mobilen Plattformen finden sich in Kapitel 2.3)

Die genauen Zahlen der jeweiligen Kategorien können *Tabelle 1* entnommen werden. Bei der Betrachtung der Werte fällt auf, dass nahezu 50% der untersuchten Artefakte in die Kategorie “Text” fallen. Weiterhin fällt auf, dass Plakate mit etwa 3% aller Artefakte kaum vorhanden waren.

| Typ | Anzahl der Dateien |
| --- | ------------------ |
| Präsentation | 96 |
| Plakat | 16 |
| Interaktiv | 113 |
| Text | 223 |

Das gesammelte Material wurde im nächsten Schritt auf Fehler in der Gestaltung untersucht. Nachfolgend findet sich eine Liste der gefundenen Fehler (in absteigender Reihenfolge nach Häufigkeit ihres Auftretens):

* **Fehler bim Einsatz von Farben:** Es wurden Farbkombinationen verwendet die flimmerten, visuell anstrengende Farben wurden als Hintergrundfarbe für Texte gewählt, Unharmonische Farbkombinationen, übermäßiger Einsatz von Farben.
* **Fehlendes raster:** Beim Ausrichten der Elemente wurde kein Raster verwendet, Elemente wirken dadurch wie zufällig platziert.
* **Zu wenig Weissraum:** In der Gestaltung wurde zu wenig Whitespace verwendet, die gesamte Gestaltung ist dadurch unübersichtlich und wirkt unruhig.
* **Visuell anstrengende Tabellen:** Tabellen verwendeten mehr Trennlinien als nötig, verwendeten zu viele/unpassende Farben, es war schwer die Informationen aus der Tabelle zu entnehmen, es wurden Tabellen verwendet wo keine nötig gewesen wären.
* **Hierarchie im Text nicht deutlich:** Die Unterscheide zwischen verschiedenen Elementen(z.B. Überschriften verschiedener Ordnungen) waren zu gering, es wurde keine klare Hierarchie deutlich.
* **Fehlerhafte Ausrichtung von Elementen:** Elemente wurden in sich unsauber ausgerichtet
* **Interaktive Elemente nicht deutlich:** Es wurde nicht deutlich, mit welchen Elementen der Benutzer interagieren kann und mit welchen nicht.
* **Bilder gestreckt/gestaucht/verpixelt**
* **Inkonsistente Größen:** Gleiche Elemente waren verscheiden groß
* **Elementgrößen unverhältnismäßig:** Einige Elemente sind im Vergleich zu anderen Elementen unverhältnismäßig zu klein/zu groß
* **Zu wenig Kontrast im Text:** Text war wegen des Kontrastes schwer lesbar
* **Boxes in Boxes:** Unnötiges Verwenden von Boxen
* **Zeilenhöhe zu groß/klein**
* **Fehler im Textsatz:** Vor allem: Lücken im Blocksatz
* **Rechtschreibfehler**
* **Schlecht lesbare Schriftfamilie**

Abschließend wurde jedem der Fehler eine Oberdisziplin zugeordnet, die somit die Themengebiete definieren, die im Projekt behandelt werden sollen:

* Typographie
* Layout
* Struktur
* Farben
* Interaktive Elemente
* Tabellen
* Bilder

%----------------------------------------------------------------------------------------

\section{Kritische Reflexion}
Auch wenn dieses Fehlerorientierte vorgehen für das Projekt sehr hilfreich ist bringt es einige Gefahren mit sich.

So ist das Material von nur zwei Semestern keineswegs dazu geeignet, empirische Ergebnisse zu liefern. Die erarbeiteten Ergebnisse liefern lediglich einen Überblick über die Fehler, die in naher Vergangenheit vorherrschend waren. Außerdem ändern sich Inhalt von Modulen häufig von Semester zu Semester, eine Untersuchung im nächsten Jahr würde also mit hoher Wahrscheinlichkeit neue Fehler zeigen.

Außerdem sollten die auf Basis von Fehlern gefundenen Schwerpunkte nicht als exklusiv und vollständig angesehen werden. Es muss darauf geachtet werden, dass ggf. auch andere relevante Themen Einzug erhalten, die nicht explizit Teil der gefundenen Fehlern waren.

%----------------------------------------------------------------------------------------

\section{Mobile Plattformen}
Die beiden größten Mobilen Plattformen, *Android* und *iOS* besitzen eigene Design Guidelines, die bestimmte Themengebiete von vorn herein stark einschränken.
Als Beispiel sei hier die Wahl der Schrift genannt. Sowohl die Guidelines für Android als auch für iOS legen die Verwendung einer bestimmten Schrift nahe.
Da sich dieses Tool nicht über die Plattformspezifischen Guidelines hinwegsetzen sollte, muss von Anfang an die Plattform  auf der das Projekt des Nutzers laufen soll bekannt sein, die Inhalte des Tools müssen sich entsprechend anpassen.

%----------------------------------------------------------------------------------------

\clearpage
